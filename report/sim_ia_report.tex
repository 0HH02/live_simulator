\title{IA-Simulacion}
\author{Pepito tey}

\documentclass{article}
\begin{document}

\maketitle
\newpage

\section{Introduci\'on}
La l\'{\i}nea de este trabajo comenz\'o con la siguiente idea: ¿Que pasar\'{\i}a si pudieramos simular diferentes civilizaciones, por
decirlo de alguna manera, con diferentes ideolog\'{\i}as, tipo de gobierno, recursos, entorno. Y a la vez cada civilizaci\'on tendria
personas con diferentes formas de pensar, actuar y relacionarse, con moralidades diferentes?. ¿Y si pudieramos comparar pseudo-objetivamente
a esas civilizaciones en base a la cantidad de d\'{\i}as que sobreviven?. No fue dif\'{\i}cil darse cuenta que era una idea muy ambiciosa, y por
tanto irrealizable como tema de este trabajo. Sin embargo, esa idea sirvi\'o de inspiraci\'on para lo que finalmente se realiz\'o.

\subsection*{¿En qu\'e consiste el dilema del prisionero?}
El dilema del prisionero es un problema muy conocido en la teor\'{\i}a de juegos. Se plantea de la siguiente manera: dos personas son arrestadas por
un crimen que cometieron juntas. Cada una es interrogada por separado y se les ofrece la posibilidad de traicionar delantando a la otra persona. Si ninguno de
ellos traiciona al otro, ambos reciben una condena de 1 a\~no. Si alguno traiciona, queda libre y el otro recibe una condena de 3 a\~nos. Si ambos traicionan,
ambos reciben una condena de 2 a\~nos.

Este es solo un ejemplo de una familia de juegos llamados juegos de suma no nula o juegos de suma no cero. Pero en general los otros juegos de la misma familia
tienen la misma estructura: hay dos jugadores, cada uno se enfrenta a la decisi\'on de cooperar o traicionar al otro, y cada acci\'on produce una ganancia o
p\'erdida de recursos para cada jugador. Por supuesto, en estos juegos, la estrategia \'optima es traicionar al otro jugador, ya que es la jugada que garantiza
la mayor ganancia m\'{\i}nima. Sin embargo, lo interesante de esto, es que cambia radicalmente si se juegan m\'as rondas.

\subsection*{M\'as de dos prisioneros}
Como se mencion\'o anteriormente, este trabajo se inspir\'o en la idea de simular civilizaciones. Por lo tanto, surgi\'o la idea de simplificar esto
a un dilema del prisionero con m\'as de dos jugadores, simulando as\'{\i}, de alguna manera, una civilizaci\'on. Las reglas se abordar\'an en detalle
m\'as adelante.
\newpage

\section{Reglas del juego}
\vspace*{5mm}
\subsection*{Objetivo}
Cada jugador empieza con una cantidad inicial de recursos aleatorios(como en la vida real). Y cada d\'{\i}a pierde una cantidad fija de recursos, que pueden
representar perfectamente los recursos que gastaron ese d\'{\i}a con tal de sobrevivir. Cada jugador tendr\'a oportunidades de ganar o perder recursos al igual
que en el dilema del prisionero, la diferencia es que aqu\'{\i} cada ronda o juego se hace colaborativamente entre todos los jugadores de la civilizaci\'on.
El objetivo de cada juegador es simplemente sobrevivir el mayor n\'umero de d\'{\i}as posibles.

\subsection*{Ronda}

\subsection*{Otras mec\'anicas del juego}

\end{document}