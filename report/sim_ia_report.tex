\title{IA-Simulacion}
\author{Pepito tey}

\documentclass{article}
\begin{document}

\maketitle
\newpage
\tableofcontents
\newpage

\section{Introduci\'on}
La l\'{\i}nea de este trabajo comenz\'o con la siguiente idea: ¿Que pasar\'{\i}a si pudieramos simular diferentes civilizaciones, por
decirlo de alguna manera, con diferentes ideolog\'{\i}as, tipo de gobierno, recursos, entorno. Y a la vez cada civilizaci\'on tendria
personas con diferentes formas de pensar, actuar y relacionarse, con moralidades diferentes?. ¿Y si pudieramos comparar pseudo-objetivamente
a esas civilizaciones en base a la cantidad de d\'{\i}as que sobreviven?. No fue dif\'{\i}cil darse cuenta que era una idea muy ambiciosa, y por
tanto irrealizable como tema de este trabajo. Sin embargo, esa idea sirvi\'o de inspiraci\'on para lo que finalmente se realiz\'o.

\subsection*{¿En qu\'e consiste el dilema del prisionero?}
El dilema del prisionero es un problema muy conocido en la teor\'{\i}a de juegos. Se plantea de la siguiente manera: dos personas son arrestadas por
un crimen que cometieron juntas. Cada una es interrogada por separado y se les ofrece la posibilidad de traicionar delantando a la otra persona. Si ninguno de
ellos traiciona al otro, ambos reciben una condena de 1 a\~no. Si alguno traiciona, queda libre y el otro recibe una condena de 3 a\~nos. Si ambos traicionan,
ambos reciben una condena de 2 a\~nos.

Este es solo un ejemplo de una familia de juegos llamados juegos de suma no nula o juegos de suma no cero. Pero en general los otros juegos de la misma familia
tienen la misma estructura: hay dos jugadores, cada uno se enfrenta a la decisi\'on de cooperar o traicionar al otro, y cada acci\'on produce una ganancia o
p\'erdida de recursos para cada jugador. Por supuesto, en estos juegos, la estrategia \'optima es traicionar al otro jugador, ya que es la jugada que garantiza
la mayor ganancia m\'{\i}nima. Sin embargo, lo interesante de esto, es que cambia radicalmente si se juegan m\'as rondas.

\subsection*{M\'as de dos prisioneros}
Como se mencion\'o anteriormente, este trabajo se inspir\'o en la idea de simular civilizaciones. Por lo tanto, surgi\'o la idea de simplificar esto
a un dilema del prisionero con m\'as de dos jugadores, simulando as\'{\i}, de alguna manera, una civilizaci\'on. Las reglas se abordar\'an en detalle
m\'as adelante.
\newpage

\section{Reglas del juego}
\vspace*{5mm}
\subsection{Objetivo}
Cada jugador empieza con una cantidad inicial de recursos aleatorios(como en la vida real). Y cada d\'{\i}a pierde una cantidad fija de recursos, que pueden
representar perfectamente los recursos que gastaron ese d\'{\i}a con tal de sobrevivir. Cada jugador tendr\'a oportunidades de ganar o perder recursos al igual
que en el dilema del prisionero, la diferencia es que aqu\'{\i} cada ronda o juego se hace colaborativamente entre todos los jugadores de la civilizaci\'on.
El objetivo de cada juegador es simplemente sobrevivir el mayor n\'umero de d\'{\i}as posibles.

\subsection{Ronda}
Cada ronda del juego representa un d\'{\i}a de la civilizaci\'on, cada d\'{\i}a surge un evento nuevo que puede ser de diferentes tipos, cada evento intenta
simular de la forma m\'as parecida posible situaciones que podr\'{\i}an ser reales en una civilizaci\'on de este estilo.

\paragraph*{Eventos cooperativos:}
Estos eventos tienen asociado una cantidad positiva o negativa de recursos, ya que pueden ser eventos positivos o negativos. Estos tipos de eventos hacen alusi\'on a situaciones del estilo:
\begin{itemize}
    \item Ha aparecido una mina de carb\'on a 5km de distancia es necesario ir a minar el carb\'on.
    \item El trigo ha madurado y es necesario cosecharlo.
    \item Se necesita madera en la civilizaci\'on y se necesita ir a talar \'arboles.
    \item Es necesario crear pozos para tener un f\'acil acceso al agua potable en vez de acarrear agua desde el r\'{\i}o.
    \item Con el fin de conseguir carne se est\'an agrupando en el pueblo para ir de cacer\'{\i}a.
    \item Las fuertes lluvias han tra\'{\i}do inundaciones al pueblo, es necesario la colaboraci\'on de todos para reparar los destrozos.
\end{itemize}
Aqu\'{\i} los pobladores se dividen en grupos aleatoriamente, y empieza el juego. Cada uno debe tomar una decisi\'on, que puede ser: \textbf{cooperar},
\textbf{explotar} o \textbf{trabajar en solitario}. Luego esa decisi\'on se "enfrenta" con las decisiones de los otros integrantes de su grupo siguiendo la
siguiente matriz de decisisones y ganancias.
\vspace*{2mm}
\begin{center}
    \begin{tabular}{|l|c|c|c|}
        \hline
                 & Cooperar & Explotar & Solo  \\
        \hline
        Cooperar & (10,10)  & (0,15)   & (8,8) \\
        \hline
        Explotar & (15,0)   & (0,0)    & (3,8) \\
        \hline
        Solo     & (8,8)    & (8,3)    & (8,8) \\
        \hline
    \end{tabular}
\end{center}
Podemos pensarlo de la siguiente manera: Aparece una mina de carb\'on y van varios pobladores a minar. Una persona por si sola solo
puede extraer 8 puntos de recursos, si coopera con otra persona entre los dos pueden ser m\'as eficientes al especializarce y logran
extraer 20 puntos, que se reparten equitativamente. Si uno explota y el otro coopera, como hay uno que le interesa mas robar que trabajar,
entre los dos solo logran conseguir 15 puntos, pero estos solo se los queda el que decidi\'o explotar. Si ambos explotan, trabaja y no
se ponen de acuerdo as\'{\i} que nadie obtiene nada. Y si uno explota y el otro decide ir por su cuenta el que trabaja solo ya vimos que
solo puede obtener 8 puntos, y el que explota frustrado porque no puede robar pues trabaja poco y solo obtiene 3 puntos.

Y as\'{\i} cada jugador acumula puntos en base a la decisi\'on que tom\'o y las decisiones que tomaron cada uno de los otros jugadores.
Luego de eso los recursos que ten\'{\i}a asignado ese evento se reparten en base a la cantidad de puntos de cada jugador en su grupo.
Se hace evidente que el objetivo es obtener la mayor cantidad de puntos en los eventos positivos y perder la menor cantidad en los eventos
negativos.

\paragraph*{Eventos especiales: } estos son mas f\'aciles de comprender, son eventos que pueden afectar a uno o varios pobladores, rest\'andoles
o otorg\'andoles recursos de fñorma aleatoria. Se podr\'{\i}an comparar con situaciones del estilo:
\begin{itemize}
    \item La vaca del poblador (insertar nombre gen\'erico) tuvo un ternero, lo cual aumenta su cantidad de recursos.
    \item Un \'arbol cay\'o sobre la casa de (insertar nombre gen\'erico) debido a un fuerte viento.
    \item Este verano hubo una abuntante lluvia lo cual es bueno para los cultivos.
    \item Se esparci\'o una enfermedad que afect\'o al ganado de todo el pueblo.
\end{itemize}

\subsection{Otras mec\'anicas del juego}
Hay una serie de mec\'acnicas opcionales en este juego, que pueden darle m\'as o menos ventaja a alguna estrategia en particular.
A continuaci\'on ponemos algunas de ellas.
\begin{itemize}
    \item En principio las decisiones tomadas por cada poblador son privadas, es decir cada poblador solo conoce a lo largo del juego las decisiones
          que tomaron los que estaban en su mismo grupo en cada ronda, desconociendo lo que pas\'o en los dem\'as grupos. Opcionalmente es posible hacer las
          decisiones p\'ublicas, y que cada poblador sepa cada decisi\'on tomaron los dem\'as, simulando as\'i el chisme o la comunicaci\'on entre pobladores.
          Veremos m\'as adelante que influencia tiene esto sobre el juego.
    \item Otra mec\'anica adicional es simular la reputaci\'on de los aldeanos, en base a antiguos comportamientos, esto podr\'{\i}a ser informaci\'on
          adicional para los pobladores, ya que tienen un dato cualitativo del comportamiento de otros.
    \item Apoy\'andose en la reputaci\'on mencionada anteriormente es posible a\~nadir una especie de juicio comunitario o "polic\'ia", que se encarga de que existe una posibilidad
          cada d\'{\i}a de que las personas con muy baja reputaci\'on se les prohiba participar en eventos cooperativos. Siendo esto una forma de castigo impuesto
          por los pobladores o por una "polic\'{\i}a" hacia los que peor comportamiento tienen en el pueblo.
    \item ¿Qui\'en en la vida real no ha malinterpretado las intenciones o acciones de otras personas debido a una falta de comunicaci\'on?. Pues esta es otra de las
          mec\'anicas adicionales del juego, exites una posibilidad de la elecci\'on de un jugador sea malinterpretada por los dem\'as. Por ejemplo el jugador (insertar nombre gen\'erico)
          quiso apoyar a la construcci\'on de la plaza del pueblo, pero debido a ciertas circunstancias los dem\'as pensaron que estaba holgazaneando y rob\'andose los
          materiales dados por la iglesia para la construcci\'on. Consecuencia: su reputaci\'on baj\'o debido a eso y todos pensaron que \'el hab\'{\i}a robado.
\end{itemize}
\newpage

\section{Estrategias}
En estos juegos de suma no cero la parte m\'as importante es la estrategia que se elige para jugar. Que reglas o comportamientos va a mostrar cada jugador en cada ronda.
En el tradicional juego del dilema del prisionero y otros exite una serie de estrategias o jugadores b\'asicos, los cu\'ales mostramos a continuaci\'on.
\begin{itemize}
    \item \textbf{Always cooperate} es una estrategia b\'asica que simplemente siempre elige la opci\'on de cooperar.
    \item \textbf{Always cheat} al igual que la anterior, solo que esta siempre elige explotar.
    \item \textbf{Grudger} o rencoroso, siempre coopera hasta que lo enga\~nan, a partir de ah\'i siempre roba.
    \item \textbf{Random} es la estrategia m\'as simple, no creo ni que tenga que explicarla.
    \item \textbf{Tit for tat} es uno de los m\'as conocidos, aunque tambi\'en se conoce por los nombres de ojo por ojo, the golden rule. Simplemente empieza cooperando
          y despu\'es solo copia la anterior jugada del oponente. Aunque es mucho mejor verlo de la siguiente forma, siempre coopera y cuando le roban, el devuelve el
          golpe una sola vez, despu\'es sigue cooperando. Es una estrategia bastante eficiente a pesar de su sencillez.
    \item \textbf{Tit for two tats} es literalmente el tit for tat solo que este coopera y solo roba cuando le robas dos veces seguidas, es decir: es m\'as tolerante.
\end{itemize}
Por supuesto todas estas estrategias son del juego tradicional, aunque son f\'acilmente adaptables a nuestras reglas. Adem\'as de estos jugadores b\'asicos
hemos a\~nadido nuestras propias estrategias de juego, algunas mas inteligentes que otras. Las cu\'ales exponemos a continuaci\'on.
\begin{itemize}
    \item \textbf{Exploit and explore} es una estrategia sencilla, que sigue estas dos reglas: de vez en cuando juega aleatoriamente, el resto del tiempo juega la jugada
          que m\'as puntos le ha conseguido anteriormente.
    \item \textbf{Prophet} intenta predecir el futuro de los pr\'oximos 5 d\'{\i}as y toma la jugada que maximize sus recursos.
    \item \textbf{Detective} va tratando de deducir las estrategias de los pobladores a partir de sus jugadas conocidas, y juega lo que m\'as le conviene en base a predecir
          que jugar\'an los dem\'as dadas sus estrategias.
    \item \textbf{Judge} basa sus decisiones en la reputaci\'on de los pobladores de su grupo. No colaborar\'a con la gente que juzgue de manera negativa.
\end{itemize}
\newpage

\section{Modelando el juego}

\subsection{Modelaci\'on de los agentes}

\end{document}